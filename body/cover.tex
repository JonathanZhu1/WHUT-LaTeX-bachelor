%===============  封面  =================
\smallskip
\begin{center}

\vspace*{2.2cm}
\zhongsong{\zihao{1} 武汉理工大学毕业设计(论文)} \\
\vspace*{3.3cm}
\heiti{\zihao{2} 武汉理工本科论文\LaTeX 模板 }\\
\vspace*{5.5cm}

\zhongsong
\begin{tabular}{cc}
 \zihao{-2} 学院(系):&\underline{\makebox[7cm][c]{\zihao{-2}交通学院}} \\ 
 \\
 \zihao{-2}专业班级: & \underline{\makebox[7cm][c]{\zihao{-2}船舶与海洋工程1006班}} \\ 
 \\
 \zihao{-2}学生姓名: & \underline{\makebox[7cm][c]{\zihao{-2}曹宇}} \\ 
 \\
 \zihao{-2}指导教师: & \underline{\makebox[7cm][c]{\zihao{-2}徐海祥}} \\ 
 \\
\end{tabular} 
\end{center}
\thispagestyle{empty}
\clearpage
%=====================原创性声明===========
\begin{center}
\zihao{-2} \textbf{学位论文原创性声明}
\end{center}

本人郑重声明:所呈交的论文是本人在导师的指导下独立进行研究所取得的研究成果。除了文中特别加以标注引用的内容外,本论文不包括任何其他个人或集体已经发表或撰写的成果作品。本人完全意识到本声明的法律后果由本人承担。 
\begin{flushright}
\zihao{4} 作者签名:\qquad ~~~\\

年\qquad 月\qquad 日
\end{flushright}
\vskip 2cm
\begin{center}
\zihao{-2} \textbf{学位论文版权使用授权书}
\end{center}

本学位论文作者完全了解学校有关保障、使用学位论文的规定,同意学校保留并向有关学位论文管理部门或机构送交论文的复印件和电子版,允许论文被查阅和借阅。本人授权省级优秀学士论文评选机构将本学位论文的全部或部分内容编入有关数据进行检索,可以采用影印、缩印或扫描等复制手段保存和汇编本学位论文。\smallskip

本学位论文属于
\begin{tabular}[t]{l}
1、保密$ \Box$,在~~~年解密后适用本授权书  \\ 
2、不保密$ \Box$  \\ 
\end{tabular} \\
\begin{center}
(请在以上相应方框内打“$\surd”$)
\end{center}
\begin{flushright}
\zihao{4} 作者签名:  \quad\quad\quad\quad 年 \quad  月  \quad  日\\
导师签名:   \quad\quad\quad\quad 年 \quad  月 \quad   日\\
\end{flushright}
\thispagestyle{empty}
\clearpage

%%=============设计(论文)任务书===========
%\begin{center}
%\zihao{-2}\textbf{\songti 本科生毕业设计(论文)任务书} 
%\end{center}
%\smallskip
%\renewcommand{\arraystretch}{1.3}
%\begin{tabular}{lll}
%\zihao{4} \textbf{\songti 学生姓名: 曹宇} & & \zihao{4} \textbf{\songti 专业班级:\quad\quad 船海1006班} \\ 
%\zihao{4} \textbf{\songti 指导教师:徐海祥}&\makebox [3cm] & \zihao{4} \textbf{\songti 工作单位:\quad 武汉理工大学} \\ 
%\end{tabular}\\
%\begin{tabular}{lll}
%\zihao{4} \textbf{\songti 设计(论文)题目:}& \zihao{4} \textbf{\songti  武汉理工本科论文\LaTeX 模板 } &\\ 
%\zihao{4} \textbf{\songti 设计(论文)主要内容:} \\
%\end{tabular} \\ 
%\begin{enumerate}
%\item \LaTeX 环境的配置
%\item 主要字体的控制和数学公式的选用
%\item 图表和代码的粘贴
%\end{enumerate}
%\begin{tabular}{ll}
%\zihao{4} \textbf{\songti 要求完成的主要任务:}
%\end{tabular} \\ 
%\begin{enumerate}
%\item 选择合适的\TeX 编辑系统
%\item 学习如何使用控制代码完成排版
%\item 合理的安排学习和科研的时间来发展自己兴趣爱好
%\end{enumerate}
%\begin{tabular}{ll}
%\zihao{4} \textbf{\songti 必读参考资料:}
%\end{tabular}
%\begin{enumerate}
%\item \LaTeX  \quad User Manual
%\item  字体设计的艺术
%\end{enumerate}
%\begin{tabular}{lll}
%\zihao{4} \textbf{\songti 指导教师签名: }&\makebox [4cm]& \zihao{4} \textbf{\songti 系主任签名:} \\
%& & \zihao{4} \textbf{\songti 院长签名(章)}
%\end{tabular}
%\thispagestyle{empty}
%\clearpage
%%==========本科生毕业设计(论文)开题报告  =============
%\begin{center}
%\zihao{-2} \textbf{\songti 武汉理工大学}\\
%\zihao{-2} \textbf{\songti 本科生毕业设计(论文)开题报告} 
%\end{center}
%\begin{tabular}{|l|}
%\hline \rule[-2ex]{0pt}{5.5ex} \makebox[13.5cm][l]{\zihao{4} \heiti 1、目的及意义(含国内外的研究现状分析) } \\ 
%\quad \LaTeX 是国际通行的科技论文排版软件,国际上科研机构和大学都采用它写作\\
%\quad 国内著名高校都有自己的本科生\LaTeX 模板供毕业生使用\\
%\quad 但是武汉理工大学还没有本科生\LaTeX 模板可以参考\\
%\quad 人类的价值在于创造而不是索取 \\
%\hline \rule[-2ex]{0pt}{5.5ex}  \zihao{4} \heiti
%2、基本内容和技术方案\\ 
%\quad 采用GITHUB托管降低代码维护成本\\
%\quad 加入在线\TeX 编辑器的使用简介 \\
%\quad 授人以渔,注重方法和理念的引导\\
%\hline \rule[-2ex]{0pt}{5.5ex}  \zihao{4} \heiti
%3、进度安排 \\ 
%\quad 离 deadline 两个月吃喝玩乐 \\
%\quad 离 deadline 一个月吃喝玩乐 \\
%\quad 离 deadline 半个月吃喝玩乐 \\
%\quad 离 deadline 一个星期狂写论文 \\
%\hline \rule[-2ex]{0pt}{5.5ex} \zihao{4} \heiti
%4、指导教师意见 \\ 
%\quad 曹宇同学是个好同志\\
%\quad 曹宇同志是个好同学\\
%\quad 本表格是支持跨页的长表格,你可以复制上面的内容进行测试\\
%\quad 具体方法是将tabular改为 longtable然后再编译\\
%\makebox[10cm][r]指导教师签名:\\
%\makebox[12cm][r]\quad 年\quad 月\quad 日\\
%\hline 
%\end{tabular} 
%\thispagestyle{empty}
